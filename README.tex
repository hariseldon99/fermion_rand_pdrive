\documentclass[a4paper,10pt]{article}
\usepackage[utf8]{inputenc}
\usepackage{bera}
\usepackage[centertags]{amsmath}
\usepackage{amssymb}
\usepackage{mathdots}
\usepackage{empheq}
\usepackage{dsfont}
\usepackage{amsfonts}
\usepackage{hyperref}
\usepackage{listings}
\usepackage[T1]{fontenc}
\usepackage[margin=1.0in]{geometry}
\usepackage[square, comma, numbers, sort&compress]{natbib}

\newcommand{\shellcmd}[1]{\\\indent\indent\texttt{\footnotesize\$ #1}\\}

% Command "alignedbox{}{}" for a box within an align environment
% Source: http://www.latex-community.org/forum/viewtopic.php?f=46&t=8144
\newlength\dlf  % Define a new measure, dlf
\newcommand\alignedbox[2]{
% Argument #1 = before & if there were no box (lhs)
% Argument #2 = after & if there were no box (rhs)
&  % Alignment sign of the line
{
\settowidth\dlf{$\displaystyle #1$} 
    % The width of \dlf is the width of the lhs, with a displaystyle font
\addtolength\dlf{\fboxsep+\fboxrule} 
    % Add to it the distance to the box, and the width of the line of the box
\hspace{-\dlf} 
    % Move everything dlf units to the left, so that & #1 #2 is aligned under #1 & #2
\boxed{#1 #2}
    % Put a box around lhs and rhs
}
}

%opening
\title{Algorithm for periodic quantum dynamics of a disordered fermion lattice.}
\author{Analabha Roy}

\begin{document}

\maketitle

\section{\sc Introduction}
\label{sec:intro}
\textbf{Note:} For the original one-dimensional case from which these notes are taken and modified, see~\cite{myisingrand:suppl}.
This algorithm integrates the dynamics of a fermion model of a 2-dimensional $L\times L$ square lattice. For the sake of counting convenience, we have chosen to flatten the lattice in $1d$ in row-major order. Thus, the Hamiltonian is
\begin{equation} \label{H_OBC}
H(t) = -  \frac{J}{2} \sum_{i=0}^{L^2-1}   \bigg\{c^{\dagger}_i c^{\dagger}_{i+1} + c^{\dagger}_i c^{\;}_{i+1} + \gamma \left(c^{\dagger}_i c^{\dagger}_{i+L} + c^{\dagger}_i c^{\;}_{i+L}\right)  + {\rm h.c.}\bigg\} 
    -  \sum_{i=0}^{L^2-1} \bigg\{h(t)+\alpha h_i\bigg\} c^{\dagger}_i c^{\;}_i \;,
\end{equation}
where $h(t)$ can be any periodic function, set to the default of $h_0\cos{\omega t}$. The quantities $h_i$ are random numbers, either uniform in the range $(-\sigma,\sigma)$, or Gaussian with a standard deviation of $\sigma$. The Hamiltonian can be written as a sum of the time-driven ordered Hamiltonian $ H_D(t)$ and a disordered Hamiltonian whose contribution is controlled by a weak perturbative term $\alpha$. Thus,
\begin{eqnarray}
H(t)&=& H_D(\gamma;t)+\alpha H_R.
\end{eqnarray}
The object is to do a general study of the model dynamics, and investigate the dynamics of a localized initial state~\cite{arnab1}. We can write the Hamiltonian in eq~\ref{H_OBC} in terms of Nambu spinors $|Psi$:
\begin{eqnarray}
H(t) &=& \Psi^\dagger \bar{H}(t)\Psi,\nonumber\\
\Psi &\equiv& \begin{pmatrix}
c_0\\
c_1\\
\vdots\\
c_{L^2-1}\\
c^\dagger_0\\
c^\dagger_1\\
\vdots\\
c^\dagger_{L^2-1}
\end{pmatrix},\nonumber \\
\bar{H}(t) &=&\begin{pmatrix}
A(t) & B(t)\\
-B(t) & A(t)
\end{pmatrix}.
\label{eq:nambu}
\end{eqnarray}
The submatrices $A(t)$ and $B(t)$ are real and of size $L^2\times L^2$. The matrix $A=A^d(t)+A^o$, where $A^d_{i,j}(t)=-\left[h(t)+\alpha h_i\right]\delta_{i,j}/2$ is diagonal. The
non-zero elements of $A^o$ and $B$ are given by $A^o_{i,i+1}=A^o_{i+1,i}=-J/4, A^o_{i,i+L}=A^o_{i+L,i}=-J\gamma/4$, $B_{i,i+1}=-B_{i+1,i}=- J/4, B_{i,i+L}=-B_{i+L,i}=- J\gamma/4$.

The Heisenberg quantum dynamics of the operator $c_{i,H}(t)$ are
\begin{equation}
\label{eq:heisenberg}
i \frac{\mathrm{d}}{\mathrm{d} t}c^{\;}_{i,H}(t) = U^\dagger(t)\left[c_i,H(t)\right]U(t),
\end{equation}
where the subscript $H$ denotes operators in the Heisenberg picture, and  $U(t)$ is the time translation that solves the Schr\"odinger equation of the Hamiltonian in Eq.~\ref{H_OBC}. Note that, for every operator $O(t)$ in the Schr\"odinger picture, the corresponding operator in the Heisenberg picture, $O_H(t)=U^\dagger(t) O(t) U(t)$. We now substitute the last of Eq.~\ref{eq:nambu} into the rhs of Eq.~\ref{eq:heisenberg} to yield
\begin{equation}
\label{eq:dyn}
i \frac{\mathrm{d}}{\mathrm{d} t}c^{\;}_{i,H}(t) =  2\sum_{j=0}^{L^2-1} \bigg[A_{ij}(t)c^{\;}_{j,H}(t)+ B_{ij}(t)c^{\dagger}_{i,H}(t)\bigg].
\end{equation}
These equations can be solved by the trial solution $c^{\;}_{i,H}(t) = \sum_{\mu=0}^{L^2-1} \left[u_{i\mu}(t)\gamma_{\mu,H}(t)+v^\ast_{i\mu}(t)\gamma^\dagger_{\mu,H}(t)\right]$, where the Bogoliubov ansatz makes the many body state $|\Psi(t)\rangle$ the emergent vacuum of the modes  created by $\gamma^\dagger_{\mu}(t)$ $\forall t$ \textit{viz.} $|\Psi(t)\rangle \sim \prod_{\mu=0}^{L^2-1} \gamma_\mu(t) |0\rangle$, where the normalization has been suppressed for brevity. A necessary and sufficient condition for the emergent modes to generate the emergent vacuum is that $\gamma_\nu(t)|\Psi(t)\rangle = 0$, leading to  $ {d\over dt} \left( \gamma_\nu(t)|\Psi(t)\rangle \right) = 0$. In the Heisenberg picture, this condition can be rewritten as $i{d\over dt}\gamma_{\mu,H}(t)=0$. Using this result, we substitute the trial solution into Eq.~\ref{eq:dyn} and compare operator coefficients. This leads to the following dynamical system~\cite{isingrand, myisingrand}
\begin{empheq} [box=\fbox]{align}
\label{BdG_tdep:eqn}
i\frac{d}{dt}u_{i\mu}(t) &=  
{2} \sum_{j=0}^{L^2-1} \left[A_{i,j}(t)u_{j\mu}(t)+B_{i,j}(t)v_{j\mu}(t) \right] 
\nonumber \\
i\frac{d}{dt}v_{i\mu}(t) \!\! &= 
-{2}\sum_{j=0}^{L^2-1} \left[A_{i,j}(t)v_{j\mu}(t)+B_{i,j}(t)u_{j\mu}(t) \right].
\end{empheq}
Here $A$ and $B^o$ are real $L^2\times L^2$ matrices.  Furthermore, the state of the system is characterized by two \textit{complex} $L^2\times L^2$ matrices $u$ and $v$, whose columns are the $L^2$-dimensional vectors $u_\mu$ and $v_\mu$ for $\mu=0,\dots,L^2-1$. Unitary evolution and canonicality demands that $\{\gamma_\mu(t),\gamma^\dagger_\nu(t)\}=\delta_{\mu\nu},\{\gamma_\mu(t),\gamma_\nu(t)\}=\{\gamma^\dagger_\mu(t),\gamma^\dagger_\nu(t)\}=0$. Thus, $u^\dagger(t) u(t) + v^\dagger(t) v(t) = \mathds{1}$, and $v^T(t)u(t)+u^T(t)v(t) = 0$.  Any initial condition we choose must satisfy these conditions at $t=0$. Our choice of initial condition is a typical state where each lattice site is occupied by one fermion. Thus,
$|\Psi(0)\rangle \sim \prod_{i=0}^{L^2-1} c^\dagger_i |0\rangle$. This can be set by putting $u(0)=0$, $v(0)=\mathds{1}$.

Note that, for $\alpha=0.0$, the system is perfectly ordered and evolves in time as a driven fermion model along the lines of~\cite{arnab1}. 

\section{Calculation of Dynamical Responses to the Drive}
\label{sec:responses}
We shall now use these results to obtain expressions for the measurable responses of the system as it evolves in time, namely the  Fermion density, and the autocorrelations of the system. 

\subsection{Fermion Density}
\label{subsec:magcalc}
The fermion density is~\cite{arnab1} 
\begin{equation}
\label{eq:magdef}
\rho_f(t)\equiv  \frac{1}{L^2}\sum^{L^2-1}_{i=0} \langle \psi(0) | c^\dagger_{i,H} (t) c_{i,H}(t) |\psi(0)\rangle,
\end{equation}
where the operators 
\begin{equation}
\label{eq:cit}
c_{i,H}(t)\equiv \sum^{L^2-1}_{\mu=0} \left[u_{i\mu}(t)\gamma_\mu+v^\ast_{i\mu}(t)\gamma^\dagger_\mu\right],
\end{equation}
are in the Heisenberg picture, and $|\psi(0)\rangle$ is the initial state at $t=0$. Note that the number operator is NOT conserved by the Hamiltonian in eq~\ref{H_OBC}. Equation~\ref{eq:magdef} can be expanded using eq~\ref{eq:cit} to yield
\begin{eqnarray}
\label{eq:magops}
 \rho_f(t) &=& \langle\psi(0)|\hat{\rho}(t)|\psi(0)\rangle,\nonumber \\
 \hat{\rho}(t) &=& \frac{1}{L^2}\sum^{L^2-1}_{i,\mu,\nu = 0}\bigg[u^\ast_{i\nu}(t) u_{i\mu}(t)\gamma^\dagger_\nu\gamma_\mu + u^\ast_{i\nu}(t)v^\ast_{i\mu}(t)\gamma^\dagger_\nu\gamma^\dagger_\mu + \nonumber \\
  & & v_{i\nu}(t)u_{i\mu}(t)\gamma_\nu\gamma_\mu + v_{i\nu}(t)v^\ast_{i\mu}(t)\gamma_\nu\gamma^\dagger_\mu\bigg] .
\end{eqnarray}
The expression for $\hat{\rho}(t)$ can be written as
\begin{equation}
 \label{m:t}
 \hat{\rho}(t) = \frac{1}{L^2} \times \rm{Tr}\left[\hat{\kappa^i} \; u^\dagger(t)u(t) + \hat{\kappa^c} \; v^\dagger(t)v(t) + \hat{\epsilon^i} \; u^\dagger(t)v^\ast(t)+\hat{\epsilon^c} \; v^{\ast \dagger}(t)u(t)\right],
\end{equation}
where the matrix elements of $\hat{\rho^{i,c}}$ and $\hat{\epsilon^{i,c}}$ are
\begin{eqnarray}
\label{matelems}
\hat{\kappa^i}_{\nu\mu} &=& \gamma^\dagger_\nu\gamma_\mu, \nonumber \\
\hat{\epsilon^i}_{\nu\mu} &=& \gamma^\dagger_\nu\gamma^\dagger_\mu,\nonumber \\
\hat{\kappa^c}_{\nu\mu} &=& \gamma_\nu\gamma^\dagger_\mu, \nonumber \\
\hat{\epsilon^c}_{\nu\mu} &=& \gamma_\nu\gamma_\mu.
\end{eqnarray}
Now, note that, all but the third equation among these are normal ordered. Also, since $\gamma_\mu$ ($\gamma^\dagger_\mu$) destroys $|\psi(0)\rangle$ ($\langle\psi(0)|$), all but $\hat{\kappa^c}_{\nu\mu}$ among the operators in eqs.~\ref{matelems} have zero expectation values w.r.t $|\psi(0)\rangle$. This allows for the simplification of eq~\ref{m:t} and then eq~\ref{eq:magops}, yielding
\begin{equation}
\rho_f(t) = \frac{1}{L^2} \times {\rm Tr}\left[\langle\kappa^c\rangle {v}^\dagger(t) {v}(t)\right],
\end{equation}
where the expectation $\langle\dots\rangle\equiv \langle\psi(0)|\dots|\psi(0)\rangle$. Now, noting that the fermionic nature of the $\gamma$s  demands that $\hat{\kappa^c}_{\nu\mu}\equiv\gamma_\nu\gamma^\dagger_\mu=\delta_{\nu\mu}-\gamma^\dagger_\nu\gamma_\mu$, and taking expectation on both sides yields $\langle\kappa^c\rangle=\mathds{1}$. This gives us the working formula for the fermion density
\begin{equation}
 \label{eq:mag}
\boxed{\rho_f(t) = \frac{1}{L^2} \times {\rm Tr}\left[ {v}^\dagger(t) {v}(t)\right].}
\end{equation}

\section{\sc Notes on Algorithm}
The code in this tarball integrates the dynamics above numerically in a parallel multithreaded computing environment. It evaluates the magnetization and entanglement as functions of time, and dumps the output files. Here are some preliminary notes and suggestions on the algorithm used to implement the dynamics described in the previous section.
\begin{itemize}
 \item
 The source code requires the following dependencies to compile successfully
 \begin{enumerate}
  \item 
  A GNU shell environment with the GNU bash shell. On most UNIX and Linux systems, this is installed by default. On windows and macs, please install the MinGW shell environment~\cite{mingw} or Cygwin~\cite{cygwin}.
  \item
  A GNU - compatible C compiler and linker. On most UNIX and Linux systems, the GNU-CC (gcc) compiler an be easily installed. On windows and macs, please install such a compiler in your MinGW or Cygwin installations~\cite{gccmingw,gcccygwin}. Any GNU compatible C compiler should do it, whether its gcc or any other like icc (Intel C compiler), bcc (Bourne C compiler), etc.
  \item
  The GNU Make toolkit~\cite{make}. On most UNIX and Linux systems, this is installed by default. On windows and macs, please install GNU Make in your MinGW or Cygwin installations in a manner similar to~\cite{gccmingw,gcccygwin}.
  \item
  \LaTeX , for the documentation.
  \item
  The \LaTeX - autocompiler 'latex-mk'~\cite{latexmk}.
  \item
  The GNU Scientific library~\cite{galassi:gsl}. This is required for the matrix implementation, the default BLAS (Basic Linear Algebra Subprograms)~\cite{blas} for matrix-matrix multiplications, and the ODE integrators. You can link other BLAS libraries if you want, and boiler plate changes to the code will not be necessary. In addition, the uniform random number generators of the GSL are being used.
  \item
  The GLib library and associated header file 'glib.h'~\cite{glib}. This is a cross-platform software utility library that  provides advanced data structures, such as linked lists, hash tables, dynamic arrays, balanced binary trees etc. The dynamic array type from this library will be used to store output data during runtime.
  \item
  Any implementation of the MPI standard of Message Passing Parallelization. For details, see~\cite{mpi}. The program has been tested with OpenMPI~\cite{openmpi}.
\item
 Optional: The Python Programming Language~\cite{python}, as well as numpy~\cite{numpy}, scipy~\cite{scipy} and matplotlib~\cite{matplotlib} packages for the postprocessing scripts.
 \end{enumerate}
 \item
 The code is in a 'tarball' that can be uncompressed using any decompression tool like GNU tar, or 7-zip. The source code is distributed in the C files, and default makefiles are provided. Please adjust the makefile as needed before compiling. The tarball also provides scripts for running the compiled binary.
 \item
 To compile the code, just untar the package and run  
 \shellcmd{make}
 This should compile the code into a single binary named 'isingrand\_parallel'. Executing this without any flags will dump out usage instructions.  
 \item
 To compile any particular object, like the integrator or the main file, simply run 'make' followed by the object name. For example, to compile the integrator object, run
 \shellcmd{make integrator.o}
 This will create the object file 'integrator.o'.
 \item
 To build the documentation from \LaTeX , simply navigate to the 'writeup' directory and run
 \shellcmd{make dvi/pdf/ps/html}
 Choose any one of the above options. This will build the document from the \LaTeX - file.
 \item
  I am implementing the integrators in the GNU Scientific Library~\cite{galassi:gsl} for solving the actual dynamics. The library contains many implementations of Runge Kutta and Bulirsch St\"oer routines that can be used and interchanged easily. However, using Bulirsch St\"oer routine requires the calculation of the Jacobian of the dynamics, and I have not implemented this currently. The jacobian is currently just a placeholder blank function that returns the macro GSL\_SUCCESS without actually doing anything. Runge Kutta methods will work, and the default method coded is the 'rk8pd' method \textit{i.e.} the $8^{th}$ order Runge-Kutta Prince Dormand method with $9^{th}$ order error checking.
\item
 The program runs multiple instances of the random number generation via a loop that runs through $N$ counts of the random number generator. This loop is parallelized.
\item
 The program uses MPI programming to distribute the work load across multiple processors in multiple nodes of a cluster.  The program can be run with $p$ MPI processes. Each mpi process generates its own seed, instantiates its own copy of a random number generator using that seed.
\item
In order to run the program with only one processor in one node, just run the compiled binary with no arguments to see a help page with a list of all options and flags. To run it in a standard mpi environment, run the binary and its options/flags after appending them to an MPI runtime command like 'mpiexec','mirun','ibrun','poe' etc. Sample run scripts are given in the 'scripts' directory.
\item
The 'scripts' directory also contains some python scripts to postprocess the data from the main program. They have the extension '.py' and are individually commented in detail.
  \item 
  I have \textbf{not hardcoded} the input block except for the lattice size, and I strongly suggest keeping it that way. The data is read using the 'getopt' library in glibc~\cite{getopt}, and there is a sample run script named 'runprog.sh' that runs the program with default values. Please alter it as necessary.
 \item
 To speed up program execution, output data is dumped to a dynamic array in memory. After the integrations, the array data is dumped to disk. This is faster than dumping to disk during actual runtime, although this does mean that the program eats up a lot of memory. The dynamic arrays are constructed using data collection tools from the GLib library~\cite{glib}, such as the 'Array' data structure. This method guards against memory leaks that may arise from incorrectly using \textbf{malloc()} or \textbf{realloc()} directly.
\end{itemize}

\begin{thebibliography}{10}
\bibitem{myisingrand}
\newblock A. Roy and A. Das, Phys. Rev. B {\bf 91}, 121106(R), (2015).
\bibitem{myisingrand:suppl}
See Supplemental Material of~\cite{myisingrand}.
\bibitem{isingrand}
\newblock T. Caneva, R. Fazio, and G.E. Santoro, Phys. Rev. B {\bf 76}, 144427 (2007).
\newblock DOI : \url{http://link.aps.org/doi/10.1103/PhysRevB.76.144427}.
\bibitem{arnab1} Das A 2010 {Phys. Rev. B} {\bf 82} 172402. Bhattacharyya S, Das A
and Dasgupta S 2012 {Phys. Rev. B} {\bf 86} 054410.
\bibitem{mingw}
MinGW, a contraction of "Minimalist GNU for Windows", is a minimalist development environment for native Microsoft Windows applications.
Please see \url{http://www.mingw.org/}.
\bibitem{cygwin}
Cygwin is a collection of tools which provide a Linux look and feel environment for Windows. Please see \url{http://www.cygwin.com/}.
\bibitem{gccmingw}
\newblock To install gcc on MinGW, install MinGW~\cite{mingw}, then install the cli addons installer from \url{http://sourceforge.net/projects/mingw/files/Installer/mingw-get/}, and run
\shellcmd{mingw-get install gcc g++ mingw32-make}
Also see \url{http://www.mingw.org/wiki/Getting_Started}. You can also play with this \url{http://mingw-w64.sourceforge.net/} for $64-$ bitness.
\bibitem{gcccygwin}
\newblock To install gcc on cygwin, run the cygwin installer \url{http://www.cygwin.com/install.html} and choose gcc in the menu. 
\bibitem{make}
GNU Make is a tool which controls the generation of executables and other non-source files of a program from the program's source files. See
\url{http://www.gnu.org/software/make/}
\bibitem{latexmk}
LaTeX-Mk is an automatic \LaTeX - compiler that uses makefiles~\cite{make} to build \LaTeX - files. See \url{http://latex-mk.sourceforge.net/}.
\bibitem{blas}
Basic Linear Algebra Subroutine (BLAS) is an API standard for building linear algebra libraries that perform basic linear algebra operations such as vector and matrix multiplication. See \url{http://www.netlib.org/blas/}. For a quick introduction, see \url{http://www.netlib.org/scalapack/tutorial/sld054.htm}.
\bibitem{galassi:gsl}
M.Galassi, J.~Davies, J.~Theiler, B.~Gough, G.~Jungman, M.~Booth, and F.~Rossi, 
{\em GNU Scientific Library Reference Manual,  2nd edition},
(Network Theory Ltd.,Bristol BS8 3AL, United Kingdom, 2003).
\newblock Website:\url{http://www.gnu.org/software/gsl/manual/gsl-ref.html}

\bibitem{gsl:qrdecomp}
The QR decomposition of a matrix and the extraction of the decomposed products can be done numerically using the GNU Scientific Library. The relevant documentation can be found at~\url{http://www.gnu.org/software/gsl/manual/html_node/QR-Decomposition.html}.


\bibitem{mpi}
\newblock For a quick introduction to parallel computing and the message passing interface, see \\
\newblock V. Eijkhout,
{\em Introduction to High-Performance Scientific Computing},
\newblock Web:\url{http://tacc-web.austin.utexas.edu/veijkhout/public_html/istc/istc.html}.\\
For details, see \\
\newblock W. Gropp, and E. Lusk,
{\em An Introduction to MPI: Parallel Programming with the Message Passing Interface}\\
\newblock Web:\url{http://www.mcs.anl.gov/research/projects/mpi/tutorial/mpiintro/ppframe.htm}

\bibitem{openmpi}
\newblock Open MPI: Open Source High Performance Computing
\newblock The Open MPI Project is an open source MPI-2 implementation that is developed and maintained by a consortium of academic, research, and industry partners.
\newblock: URL: \url{http://www.open-mpi.org/}.

\bibitem{python}
Python is a powerful dynamic high-level programming language similar to Tcl, Perl, Ruby, Scheme or Java. For details, see \url{http://www.python.org/}.

\bibitem{numpy}
NumPy is the fundamental package for scientific computing with Python. It implements array storage and object-oriented numerical class libraries as python modules. For details see \url{http://www.numpy.org}.

\bibitem{scipy}
The SciPy Stack is a collection of open source software for scientific computing in Python, and particularly a specified set of core packages. For details see \url{http://www.scipy.org}.

\bibitem{matplotlib}
Matplotlib is a mature and popular plotting package in Python. It provides publication-quality 2D plotting as well as rudimentary 3D plotting.For details see \url{http://www.matplotlib.org}.

\bibitem{getopt}
Getopt is a C library function used to parse command-line options. See \url{http://www.gnu.org/software/libc/manual/html_node/Getopt.html}. It is part of the glibc software package:\url{http://www.gnu.org/software/libc/}

\bibitem{glib}
\newblock GLib Reference manual:\url{https://developer.gnome.org/glib/}\\
Also see \\
\newblock T. Copeland, {\em Manage C data using the GLib collections}, (IBM:2005)
\newblock Web:\url{http://www.ibm.com/developerworks/linux/tutorials/l-glib/}

\bibitem{unitaryflow}
A. Verdeny, A. Mielke, and F. Mintert, Phys. Rev. Lett. {\bf 111}, 175301 (2013).

\bibitem{hatano}
M. Fujinaga, and N. Hatano, J. Phys. Soc. Jpn. {\bf 76} 094001 (2007).

\bibitem{osborne}
T. J. Osborne, M. A. Nielsen, Phys. Rev. A {\bf 66}, 032110 (2002).

\bibitem{lieb}
E. Lieb, T. Schultz, and D. Mattis, Ann. Phys. {\bf 16}, 407 (1961).

\bibitem{vidal}
G. Vidal and R.F. Werner, Phys. Rev. A {\bf 65}, 032314 (2002).

\bibitem{plenio}
M.B. Plenio, Phys. Rev. Lett. {\bf 95} 090503 (2005).

\bibitem{igloi}
F. Igloi and H. Reiger, Phys. Rev. B {\bf 57} (18) 11404 (1998).

\bibitem{canovi}
E. Canovi, \textit{Quench dynamics of many-body systems}, Thesis at \url{http://www.sissa.it/statistical/tesi_phd/canovi.pdf} (2010).
\end{thebibliography}


\end{document}
